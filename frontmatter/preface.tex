\chapter*{Preface}

Spatial microsimulation is a set of methods for modelling phenomena operating
at individual and geographical levels simultaneously.
The city is a classic example of a system that the techniques
presented in this book can help us understand: Shanghai
(depicted on the front cover) is a highly complex metropolis comprised
of more than 10 million interacting people. Such a system is clearly hard
to simulate. Yet by breaking the problem up into its
constituent parts --- discrete geographic areas and a sample of the human
population --- spatial microsimulation could be used to represent the city
on a standard laptop computer. Of course we should of the reductionist nature
of all models but the advantages of condensing the fundamentals of multi-level
systems down into it their constituent parts should be clear to the reader.
We could simulate population growth, increased energy efficiency or even
a ban on cars. Moreover, by linking the synthetic population created during
spatial microsimulation to an agent-based model, complexity and 'emergent'
behaviour could also be simulated.

This book is for anyone who wants to not only understand
but to \emph{use} spatial microsimulation, for applied or academic work.
The emphasis is therefore practical rather than theoretical.
New R packages for assisting with the generation of spatial microdata are described in
detail, with reference to practical examples and reproducible code.
The aim is to enable you to implement the methods on your own
datasets to better understand the world. 

Knowledge is power. By enabling the representation of the fundamentals,
this book should enhance your knowledge and understanding of large and complex
systems such as Shanhai. In combination with an understanding of social and
political realities (which may not be conducive to computer modelling)
this book should therefore also empower its readers to change the world.
Used carefully with an understanding of the decision making process,
spatial microsimulation enables evidence-based policy.
% and are best understood with
% reference to my own experience of trying to learn spatial microsimulation.

% Like many PhD students
% and researchers, I was tasked with implementing a method that
% was almost completely new to me.
% I had barely heard of Spatial Microsimulation,
% let alone understood the methodology.
% The first stage was to read up on what spatial microsimulation
% actually was: in this area I had good guidance from my
% supervisor, Dimitris Ballas, who pointed me towards an accessible introduction
% to the subject commissioned by the Joseph Roundtree Foundation
% (Ballas, 2005).
% This text explained clearly what spatial microsimulation
% was, but not how to implement it --- that I had to learn,
% from scratch.

Informing these lofty aims, the origins of this book go back to my PhD, when
I was tasked with using spatial microsimulation to model transport energy use.
There was a growing academic literature on the subject,
but seemingly nothing that explained \emph{how} to do spatial microsimulation,
using a modern programming language.
% A breakthrough came when I requested to see some of the
% R code used by Malcolm Campbell, who had
% recently completed a PhD using
% spatial microsimulation to explore health inequalities
% in Scotland.
% Malcolm's code proved invaluable, providing a basis
% on which I could write my own model.
It was informal communication and code-sharing from a colleague,
Malcolm Campbell, that led to the development of my models in R.
This experience demonstrated the importance of
reproducible research. Following this `open science' ethic,
the source code underlying this book is freely
available online. Readers are encouraged to comment and contribute to
the book's continued development via the code sharing site GitHub, where
an up-to-date version is hosted.
(see github.com/Robinlovelace/spatial-microsim-book).

% The number of
% hours spent dubugging and cursing the code would have been
% greatly reduced if I had had a reference text on the subject.
% This is the book I wish I'd had.
% The trigger that turned these ideas into a concrete book proposal was
% a two day course,
% An Introduction to Spatial Microsimulation with R,
% held at the University
% of Leeds in the spring of 2014.
The opportunity to make the dream of this book a reality came in the spring
of 2014, when I delivered a practical course on the topic at
the University of Leeds.
The high demand and positive feedback from participants
showed the need for practical teaching materials in the area.
% , it seemed there was a
% definite need for more practical teaching material on spatial microsimulation
% than was available at the time: many people had read the literature and had
% a good idea about the problem that they wanted to use the method to solve.
% Yet the majority were dissatisfied with the practical content of existing
% work, or lack thereof.
Four months later
% , on the 18$^{th}$ to 19$^{th}$ of September 2014,
I delivered another course on spatial microsimulation
at the University of Cambridge.
The materials had been greatly updated
and it became apparent that there was demand for more.

Morgane Dumont, who attended the Cambridge course,
became involved shortly after and has greatly improved the work.
Morganes's background in Mathematics and Statistics makes her the
ideal co-author, complementing the focus on practical examples
and code. Special mention is also in order for Maja Zaloznik
(University of Oxford)
and Richard Ellison (University of Sydney),
contributed chapters on
agent based models (ABM) and an approach known as TRESIS, respectively.

 % has contributed greatly to the book, including material on the core topics of the mipfp package and household allocation.

% In summary, this is the book I wish I had during my PhD and
% it should be of use to a wide range of researchers and practitioners.
% The practical nature of the content should make the content especially well-suited to educators, for example as part of a module on spatial microsimulation.
% Spatial microsimulation with R is intended to be a clear, succinct and above all useful introduction to the field.

\emph{Spatial microsimulation with R} is therefore the
result of international teamwork.
It is, to the best of our knowledge, the only practical book in the
field. We hope it is useful in your work.
More widely, we hope it provides a
a foundation for advancement in the
field and a toolkit for solving real-world problems.

If you have found any of the contents of the book useful, interesting
or even unclear, please do get in touch.

Robin Lovelace, September 2015, Leeds.


% Something on community involvement and contributing.

% The practical guidance should be applicable to a wide range of problems.

% \section{Acknowledgements}
% 
% As with any worthwhile textbook, this was not a solo effort. I benefited
% immensly from teaching spatial microsimulation to diverse audiences, the
% formal and informal feedback they provided, and correspondence with a number
% of people using spatial microsimulation `in the wild'. Of these,
% the following deserve special mention:
% 
% \begin{itemize}
%   \item James Gleeson, from the Greater London Authority (GLA), provided insight
%   into how spatial microsimulation can be used in local government and made several
%   improvements to the book.
%   \item Ulrike Rauer, from the University of Oxford, commented on
%   early drafts of the book and showed how it could be made more relevant to PhD
%   students new to the approach.
%   \item Stephen Clarke at the University of Leeds demonstrated the benefits of
%   the Flexible Modelling Framework and encouraged testing of the R code on much
%   larger datasets than had previously been used, encouraging optimisation of the code.
% \end{itemize}
% 
% Thanks also to all the people
% who provided the wider resources for this project to happen.


% Another advantage
% I had was Malcolm Campbell as a predecessor.
% Malcolm provided a huge amount of
% support during the early phase of my PhD and shared
% all the R code he developed.
% Not only was this invaluable to my efforts to
% build a spatial microsimulation model in
% R (some of the code in the book is probably his at some level);
% his example of collaboration
% and code sharing was inspirational. It is this generosity,
% displayed every day in the open source software movement,
% that drove my desire to write this book. After all, free
% to produce and communicate, knowledge is the ultimate renewable resource.
