\chapter*{Preface}

Spatial microsimulation is a set of methods for modelling phenomena which operate
at individual and geographical levels simultaneously.
For example the functioning of a modern city (Shanghai is illustrated on the front cover)
involves an overwhelmingly complex web of human interactions.
Simulating such complexity may seem impossible.
Yet, by breaking the problem up into its
constituent parts --- discrete geographic areas and a sample of the
population --- spatial microsimulation can be used to model key aspects of the city system on
an everyday laptop computer.
There are dangers associated with reductionism, but
condensing a problem down to its fundamentals has many advantages. 
Using the techniques described we can simulate scenarios such as population growth, increased energy efficiency and major shifts in transport technologies and modes.
By linking the synthetic population
to an agent-based model, in which individuals interact over time with each other and their environment, complex
behaviours such as social segregation could also be simulated, as illustrated in Chapter 12.

This book is for anyone who wants to not only understand
but to \emph{use} spatial microsimulation.
The emphasis is on the practical rather than theoretical aspects of the field.
R packages such as \textbf{mipfp}, for enabling the generation of synthetic populations, are described in
detail, with reference to practical examples and reproducible code.
The aim is to enable you to implement the methods on your own
data. 

By explaining how to use tools for modelling
phenomena that vary over space,
this book should help enhance your knowledge of
complex systems. We hope the book empowers the reader with the confidence and know-how
needed to provide evidence-based policy guidance.

% Like many PhD students
% and researchers, I was tasked with implementing a method that
% was almost completely new to me.
% I had barely heard of Spatial Microsimulation,
% let alone understood the methodology.
% The first stage was to read up on what spatial microsimulation
% actually was: in this area I had good guidance from my
% supervisor, Dimitris Ballas, who pointed me towards an accessible introduction
% to the subject commissioned by the Joseph Roundtree Foundation
% (Ballas, 2005).
% This text explained clearly what spatial microsimulation
% was, but not how to implement it --- that I had to learn,
% from scratch.

The origins of this book are more prosaic: during my PhD at the University of Sheffield
I was tasked with using spatial microsimulation to model transport energy use.
Despite the growing academic literature on the subject,
there was little information that explained \emph{how} to do spatial microsimulation,
using a modern programming language.
% A breakthrough came when I requested to see some of the
% R code used by Malcolm Campbell, who had
% recently completed a PhD using
% spatial microsimulation to explore health inequalities
% in Scotland.
% Malcolm's code proved invaluable, providing a basis
% on which I could write my own model.
It was informal communication and code-sharing with a colleague,
Malcolm Campbell, that led to the development of my models in R.
This experience demonstrated the importance of
reproducible research. Following this `open science' ethic,
readers are encouraged to comment on and contribute to
the book's continued development via the code sharing site GitHub
(see \url{https://github.com/Robinlovelace/spatial-microsim-book}).

% The number of
% hours spent dubugging and cursing the code would have been
% greatly reduced if I had had a reference text on the subject.
% This is the book I wish I'd had.
% The trigger that turned these ideas into a concrete book proposal was
% a two day course,
% An Introduction to Spatial Microsimulation with R,
% held at the University
% of Leeds in the spring of 2014.
The opportunity to turn the idea into reality came in the spring
of 2014, when I developed notes for an `Introduction to Spatial Microsimulation' course at
the University of Leeds.
The high demand for and positive feedback after the course
suggested the need for practical teaching materials in the area.
% , it seemed there was a
% definite need for more practical teaching material on spatial microsimulation
% than was available at the time: many people had read the literature and had
% a good idea about the problem that they wanted to use the method to solve.
% Yet the majority were dissatisfied with the practical content of existing
% work, or lack thereof.
Four months later
% , on the 18$^{th}$ to 19$^{th}$ of September 2014,
I delivered another course on spatial microsimulation
at the University of Cambridge.
The materials had been greatly updated
and, thanks to the involvement of CRC Press, these
provided the foundation for a book on the subject.

Morgane Dumont (NaXys, University of Namur), who attended the Cambridge course,
became involved shortly after and has greatly improved the work.
Morgane's background in Mathematics and Statistics made her the
ideal co-author, complementing the focus on practical examples
and code.

Maja Zaloznik
(University of Oxford)
and Richard Ellison (University of Sydney) have also greatly improved the book
through their contributed chapters.
Richard's chapter (11) illustrates how R can be used as the basis for transport
demand modelling, using an approach known as TRESIS. Maja's chapter (12)
is the most advanced in the book and demonstrates how spatial microsimulation
can be used in parallel with agent-based modelling, with an implementation in the
NetLogo language.

 % has contributed greatly to the book, including material on the core topics of the mipfp package and household allocation.

% In summary, this is the book I wish I had during my PhD and
% it should be of use to a wide range of researchers and practitioners.
% The practical nature of the content should make the content especially well-suited to educators, for example as part of a module on spatial microsimulation.
% Spatial microsimulation with R is intended to be a clear, succinct and above all useful introduction to the field.

\emph{Spatial microsimulation with R} is therefore the
result of international teamwork.
It is, to the best of our knowledge, the only practical book on the subject.
We hope it is useful in your work.
More widely, we hope it provides
a solid foundation for advancement in the
field and a toolkit for solving real-world problems.

If you have any feedback on the book please do get in touch via the book's online repository, hosted on the code sharing platform GitHub: https://github.com/Robinlovelace/spatial-microsim-book.

Robin Lovelace, February 2016.


% Something on community involvement and contributing.

% The practical guidance should be applicable to a wide range of problems.

 \section{Acknowledgements}
 
 As with any worthwhile textbook, this was not a solo effort. We benefited
 immensely from teaching spatial microsimulation to diverse audiences, the
 formal and informal feedback they provided, and correspondence with a number
 of people using spatial microsimulation `in the wild'. Of these,
 the following deserve special mention:
 
 \begin{itemize}
   \item James Gleeson, from the Greater London Authority (GLA), provided insight
   into how spatial microsimulation can be used in local government and made several
   improvements to the book.
   \item Ulrike Rauer, from the University of Oxford, commented on
   early drafts of the book and showed how it could be made more relevant to PhD
   students new to the approach.
   \item Stephen Clarke at the University of Leeds demonstrated the benefits of
   the Flexible Modelling Framework and encouraged testing of the R code on much
   larger datasets than had previously been used, encouraging optimisation of the code.
   \item Johan Barthélemy, from the SMART Infrastructure (Wollongong), helped in
   understanding his methods and R package (mipfp).
   \item Lex Comber, who provided crucial comments on the structure of the first part of the book and a great insight into how to make it more useful for teaching.
   \item Malcolm Campbell, my predecessor in the PhD. Malcolm provided a huge amount of support during the early phase of my PhD and shared
all the R code he developed. He's been a great support of the book from the beginning.
   \item Everyone who provided input from the University of Leeds, including Mark Birkin, Nick Malleson and Andy Evans.
 \end{itemize}
 
 Morgane Dumont's research are funded by the Wallonia Region (Belgium) and she 
 is member of NaXys (University of Namur). She thanks particularly these two affiliations.
 Finally, computational resources (for Chapter 10) have been provided by the Consortium des Équipements de 
 Calcul Intensif (CÉCI), funded by the Fonds de la Recherche Scientifique de Belgique 
 (F.R.S.-FNRS) under Grant No. 2.5020.11.
 
 Thanks also to all the people
 who provided the wider resources for this project to happen.


% Another advantage
% I had was Malcolm Campbell as a predecessor.
% Malcolm provided a huge amount of
% support during the early phase of my PhD and shared
% all the R code he developed.
% Not only was this invaluable to my efforts to
% build a spatial microsimulation model in
% R (some of the code in the book is probably his at some level);
% his example of collaboration
% and code sharing was inspirational. It is this generosity,
% displayed every day in the open source software movement,
% that drove my desire to write this book. After all, free
% to produce and communicate, knowledge is the ultimate renewable resource.
